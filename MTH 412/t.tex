\documentclass[12pt]{article}
\usepackage{amsmath, amssymb, amsthm}
\usepackage{geometry}
\usepackage{graphicx}
\usepackage{float}

\geometry{margin=1in}

\title{MTH 411: Theory of Ordinary Differential Equation Assignment}
\author{OKWHAROBO Solomon Monday}
\date{\today}

\begin{document}

\maketitle

\thispagestyle{empty} % No page number for this page

\begin{center}
    \vspace{6cm}
    \textbf{Anchor University} \\
    \textbf{Department of Mathematics} \\
    \textbf{AUL/SCI/20/00605}
\end{center}





\newpage

\setcounter{page}{1} % Start page numbering from here

\section{Q1:}
\textbf{Theorem:}
Suppose $f$ is continuous on an open set $U \subset \mathbb{R} \times \mathbb{R}^n$. Let $(t_0, x_0) \in U$, and $\phi$ be a function defined on an open set $I$ of $\mathbb{R}$ such that $t_0 \in I$. Then $\phi$ is a solution of the IVP (1.3) if, and only if,

\begin{enumerate}
    \item $\forall t \in I$, $(t, \phi(t)) \in U$.
    \item $\phi$ is continuous on $I$.
    \item $\forall t \in I$, $\phi(t) = x_0 + \int_{t_0}^{t} f(s, \phi(s)) \, ds$.
\end{enumerate}

\textbf{Proof:}

\noindent (\textbf{$\Rightarrow$}) Let us suppose that $\phi' = f(t, \phi)$ for all $t \in I$ and that $\phi(t_0) = x_0$. Then for all $t \in I$, $(t, \phi(t)) \in U$ (i). Also, $\phi$ is differentiable and thus continuous on $I$ (ii). Finally,
\[\phi'(s) = f(s, \phi(s))\]
so integrating both sides from $t_0$ to $t$,
\[\phi(t) - \phi(t_0) = \int_{t_0}^{t} f(s, \phi(s)) \, ds\]
and thus
\[\phi(t) = x_0 + \int_{t_0}^{t} f(s, \phi(s)) \, ds\]
hence (iii).

\noindent (\textbf{$\Leftarrow$}) Assume i), ii), and iii). Then $\phi$ is differentiable on $I$, and $\phi'(t) = f(t, \phi(t))$ for all $t \in I$.
From (iii), $\phi(t_0) = x_0 + \int_{t_0}^{t} f(s, \phi(s)) \, ds = x_0$.




\section{Q2:}

\textbf{Claim:} If \( f \in C^k \), then all solutions \( \phi \) of the differential equation \( \frac{dx}{dt} = f(t, x) \) are of class \( C^{k+1} \).

\textbf{Proof:}

Assume \( f \) is \( C^k \), and let \( \phi(t) \) be a solution to the given differential equation on some open interval \( I \) containing \( t_0 \). We want to show that \( \phi \) is \( C^{k+1} \) on \( I \).

\textbf{1. Existence of Derivatives:}
Since \( \phi \) is a solution, it is differentiable. We want to show that \( \phi' \) (the first derivative) exists and is continuous.

\textbf{2. Differentiability:}
Differentiate the given differential equation with respect to \( t \):
\[ \frac{d}{dt}\left(\frac{dx}{dt}\right) = \frac{d}{dt}f(t, \phi(t)) \]
This gives us \( \frac{d^2\phi}{dt^2} = \frac{\partial f}{\partial t} + \frac{\partial f}{\partial x}\frac{dx}{dt} \).

\textbf{3. Continuity of Derivatives:}
Since \( f \) is \( C^k \), its partial derivatives up to order \( k \) are continuous. Also, \( \frac{dx}{dt} = f(t, \phi(t)) \) implies that \( \frac{dx}{dt} \) is continuous.

\textbf{4. Inductive Step:}
Repeat the process for higher-order derivatives. In general, you will have:
\[ \frac{d^{k+1}\phi}{dt^{k+1}} = \text{(a sum of terms involving derivatives of } f \text{ and } \phi \text{ up to order } k) \]
Again, continuity follows from the continuity of partial derivatives of \( f \) and lower-order derivatives of \( \phi \).

Therefore, by induction, all derivatives of \( \phi \) up to order \( k+1 \) exist and are continuous, implying that \( \phi \) is of class \( C^{k+1} \).

This completes the modified proof, specifically tailored for the differential equation \( \frac{dx}{dt} = f(t, x) \).

\end{document}
